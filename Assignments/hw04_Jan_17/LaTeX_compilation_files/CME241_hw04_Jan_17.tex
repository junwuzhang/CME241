\documentclass[12pt]{article}
 
\usepackage[margin=1in]{geometry}
\usepackage{amsmath,amsthm,amssymb}
\usepackage{commath}
\usepackage{graphicx}
\usepackage{float}
\usepackage{caption}
\usepackage{subcaption}
\usepackage{hyperref}
\usepackage{gensymb}
\usepackage{xparse,mathtools}
\usepackage{color,soul}

%%% --------- packages for glossary ------- %%%
\usepackage[acronym,nomain,nonumberlist]{glossaries}
\makeglossaries
%
\newacronym{LHS}{LHS}{left-hand side}
\newacronym{RHS}{RHS}{right-hand side}
\newacronym{CARA}{CARA}{Constant Absolute Risk-Aversion}
\newacronym{CRRA}{CRRA}{Constant Relative Risk-Aversion}

\usepackage[backend=bibtex,style=numeric]{biblatex}
% Select the bibliography file
\addbibresource{references.bib}

\ExplSyntaxOn

\NewDocumentCommand \vect { s o m }
 {
  \IfBooleanTF {#1}
   { \vectaux*{#3} }
   { \IfValueTF {#2} { \vectaux[#2]{#3} } { \vectaux{#3} } }
  ^T
 }

\DeclarePairedDelimiterX \vectaux [1] {\lbrack} {\rbrack}
 { \, \dbacc_vect:n { #1 } \, }

\cs_new_protected:Npn \dbacc_vect:n #1
 {
  \seq_set_split:Nnn \l_tmpa_seq { , } { #1 }
  \seq_use:Nn \l_tmpa_seq { \enspace }
 }
\ExplSyntaxOff
 
\newcommand{\N}{\mathbb{N}}
\newcommand{\R}{\mathbb{R}}
\newcommand{\Z}{\mathbb{Z}}
\newcommand{\Q}{\mathbb{Q}}

\newcommand{\myequation}{\begin{equation}}
\newcommand{\myendequation}{\end{equation}}
\let\[\myequation
\let\]\myendequation
 
\newenvironment{theorem}[2][Theorem]{\begin{trivlist}
\item[\hskip \labelsep {\bfseries #1}\hskip \labelsep {\bfseries #2.}]}{\end{trivlist}}
\newenvironment{lemma}[2][Lemma]{\begin{trivlist}
\item[\hskip \labelsep {\bfseries #1}\hskip \labelsep {\bfseries #2.}]}{\end{trivlist}}
\newenvironment{exercise}[2][Exercise]{\begin{trivlist}
\item[\hskip \labelsep {\bfseries #1}\hskip \labelsep {\bfseries #2.}]}{\end{trivlist}}
\newenvironment{problem}[2][Problem]{\begin{trivlist}
\item[\hskip \labelsep {\bfseries #1}\hskip \labelsep {\bfseries #2.}]}{\end{trivlist}}
\newenvironment{question}[2][Question]{\begin{trivlist}
\item[\hskip \labelsep {\bfseries #1}\hskip \labelsep {\bfseries #2.}]}{\end{trivlist}}
\newenvironment{corollary}[2][Corollary]{\begin{trivlist}
\item[\hskip \labelsep {\bfseries #1}\hskip \labelsep {\bfseries #2.}]}{\end{trivlist}}
\newenvironment{solution}
  {\renewcommand\qedsymbol{$\blacksquare$}\begin{proof}[Solution]}
  {\end{proof}}
 
\begin{document}
 
\title{Homework \#4}
\author{Junwu Zhang\\ 
CME 241: Reinforcement Learning for Finance}
 
\maketitle

\begin{problem}{1}
\text{ }\\
Work out (in \LaTeX) the equations for Absolute/Relative Risk Premia for CARA/CRRA respectively
\end{problem}
\begin{solution}
\text{ }\\
(1) Since risk premia are closely related to one's utility of money, calculate the risk premia, we can perform taylor-expansion on utility $U(x)$ like follows:
\begin{gather}
	U(x) \approx U(\bar{x}) + U^{\prime}(\bar{x}) \cdot (x-\bar{x}) + \frac{1}{2}U^{\prime\prime}(\bar{x})\cdot (x-\bar{x})^2 \\
	U(x_{CE}) \approx U(\bar{x}) + U^{\prime}(\bar{x}) \cdot (x_{CE} - \bar{x})
\end{gather}
Since we have $\mathbb{E}[U(x)] = U(x_{CE})$, the left hand side is the expectation of $U(x)$:
\begin{gather}
	\mathbb{E}[U(x)] \approx U(\bar{x}) + \frac{1}{2} \cdot U^{\prime\prime}(\bar{x})\cdot \sigma_x^2 = U(x_{CE}) \approx U(\bar{x}) + U^{\prime}(\bar{x}) \cdot (x_{CE} - \bar{x}) \\
	U^{\prime}(\bar{x}) \cdot (x_{CE} - \bar{x}) \approx \frac{1}{2} \cdot U^{\prime\prime}(\bar{x})\cdot \sigma_x^2
\end{gather}
Moving $U^{\prime}(\bar{x})$ to the \gls*{RHS} and $U^{\prime\prime}$ to the \gls*{LHS} of the equation, we have:
\begin{gather}
x_{CE} - \bar{x} \approx \frac{1}{2} \cdot \frac{U^{\prime\prime}(\bar{x})}{U^{\prime}(\bar{x})}\cdot \sigma_x^2 \\
\pi_A = \bar{x} - x_{CE} \approx -\frac{1}{2} \cdot A(\bar{x}) \cdot \sigma_x^2,
\end{gather}
where $A(\bar{x}) = \frac{U^{\prime\prime}(\bar{x})}{U^{\prime}(\bar{x})}$.

For \gls{CARA}, consider the utility function $U(x) = \frac{-e^{-ax}}{a}$:
\begin{gather}
	U^{\prime}(\bar{x}) = e^{-ax} \\
	U^{\prime\prime}(\bar{x}) = -a\cdot e^{-ax} \\
	\frac{U^{\prime\prime}(\bar{x})}{U^{\prime}(\bar{x})} = -a
\end{gather}

Plug it into the $\pi_A$ equation before, we have: 
\begin{equation}
\pi_A = \mu - x_{CE} = -\frac{1}{2} \cdot A(\bar{x}) \cdot \sigma_x^2 = \frac{a\sigma^2}{2}
\end{equation}
where $\mu$ is the expected value of the uncertain payment and $x_{CE}$ is the value of certainty equivalent.

Certainty equivalent refers to the amount that a person is willing to pay; in a risk-avert scenario, this value is typically lower than $\mu$. 

(2) For \gls{CRRA}, consider $U(x) = \frac{x^{1-\gamma}}{1-\gamma}$:
\begin{gather}
R(x) = \frac{-U^{\prime\prime} \cdot x}{U(x)} = \gamma
\end{gather}
When $\gamma \neq 1$:
\begin{align}
\mathbb{E}[U(x)] &= \frac{e^{\mu(1-\gamma) + \frac{\sigma^2}{2}(1-\gamma)^2}}{1-\gamma} \\
&= \frac{e^{(1-\gamma)\cdot (\mu + \frac{\sigma^2}{2}(1-\gamma))}}{1-\gamma}
\end{align}
Therefore:
\begin{align}
x_{CE} &=e^{\mu + \frac{\sigma^2}{2}(1-\gamma)} \\
\pi_R &= 1 - \frac{x_{CE}}{\bar{x}} = 1  - \frac{x_{CE}}{\frac{e^{\mu+
	\frac{\sigma^2}{2}}}{1-e^{-\frac{\sigma^2\gamma}{2}}}} \\
&= 1 - e^{-\frac{\sigma^2\gamma}{2}}
\end{align}
\end{solution}

%\begin{problem}{2-2}
%\text{ }\\
%Write the solutions to Portfolio Applications covered in class with precise notation (in \LaTeX)
%\end{problem}
%\begin{solution}
%	
%\end{solution}
%\printbibliography
\end{document}