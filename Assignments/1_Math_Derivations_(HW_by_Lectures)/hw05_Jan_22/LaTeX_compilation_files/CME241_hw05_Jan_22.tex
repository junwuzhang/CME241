\documentclass[12pt]{article}
 
\usepackage[margin=1in]{geometry}
\usepackage{amsmath,amsthm,amssymb}
\usepackage{commath}
\usepackage{graphicx}
\usepackage{float}
\usepackage{caption}
\usepackage{subcaption}
\usepackage{hyperref}
\usepackage{gensymb}
\usepackage{xparse,mathtools}
\usepackage{color,soul}
%\usepackage{datetime}

%%% --------- packages for glossary ------- %%%
\usepackage[acronym,nomain,nonumberlist]{glossaries}
\makeglossaries
%
\newacronym{LHS}{LHS}{left-hand side}
\newacronym{RHS}{RHS}{right-hand side}
\newacronym{CARA}{CARA}{Constant Absolute Risk-Aversion}
\newacronym{CRRA}{CRRA}{Constant Relative Risk-Aversion}

\usepackage[backend=bibtex,style=numeric]{biblatex}
% Select the bibliography file
\addbibresource{references.bib}

\ExplSyntaxOn

\NewDocumentCommand \vect { s o m }
 {
  \IfBooleanTF {#1}
   { \vectaux*{#3} }
   { \IfValueTF {#2} { \vectaux[#2]{#3} } { \vectaux{#3} } }
  ^T
 }

\DeclarePairedDelimiterX \vectaux [1] {\lbrack} {\rbrack}
 { \, \dbacc_vect:n { #1 } \, }

\cs_new_protected:Npn \dbacc_vect:n #1
 {
  \seq_set_split:Nnn \l_tmpa_seq { , } { #1 }
  \seq_use:Nn \l_tmpa_seq { \enspace }
 }
\ExplSyntaxOff
 
\newcommand{\N}{\mathbb{N}}
\newcommand{\R}{\mathbb{R}}
\newcommand{\Z}{\mathbb{Z}}
\newcommand{\Q}{\mathbb{Q}}

\newcommand{\myequation}{\begin{equation}}
\newcommand{\myendequation}{\end{equation}}
\let\[\myequation
\let\]\myendequation
 
\newenvironment{theorem}[2][Theorem]{\begin{trivlist}
\item[\hskip \labelsep {\bfseries #1}\hskip \labelsep {\bfseries #2.}]}{\end{trivlist}}
\newenvironment{lemma}[2][Lemma]{\begin{trivlist}
\item[\hskip \labelsep {\bfseries #1}\hskip \labelsep {\bfseries #2.}]}{\end{trivlist}}
\newenvironment{exercise}[2][Exercise]{\begin{trivlist}
\item[\hskip \labelsep {\bfseries #1}\hskip \labelsep {\bfseries #2.}]}{\end{trivlist}}
\newenvironment{problem}[2][Problem]{\begin{trivlist}
\item[\hskip \labelsep {\bfseries #1}\hskip \labelsep {\bfseries #2.}]}{\end{trivlist}}
\newenvironment{question}[2][Question]{\begin{trivlist}
\item[\hskip \labelsep {\bfseries #1}\hskip \labelsep {\bfseries #2.}]}{\end{trivlist}}
\newenvironment{corollary}[2][Corollary]{\begin{trivlist}
\item[\hskip \labelsep {\bfseries #1}\hskip \labelsep {\bfseries #2.}]}{\end{trivlist}}
\newenvironment{solution}
  {\renewcommand\qedsymbol{$\blacksquare$}\begin{proof}[Solution]}
  {\end{proof}}
 
\date{Janurary 22 , 2020}
 
\begin{document}
 
\title{Homework \#5}
\author{Junwu Zhang\\ 
CME 241: Reinforcement Learning for Finance \\}
%\date{}
\maketitle

\begin{problem}{1}
\text{ }\\
Model Merton's Portfolio problem as an MDP (write the model in \LaTeX)
\end{problem}
\begin{solution}
\text{ }\\
\textit{States} are: current wealth $W_t$

\textit{Actions} are:
\begin{equation} 
\text{Allocation of wealth:}
\begin{cases}
\text{Fraction of wealth allocated to risky assets: } \pi(t, W_t) \\
\text{Fraction of wealth allocated to riskless assets: } 1-\pi(t, W_t)
\end{cases}
\end{equation}
and wealth consumption per unit time $c(t, W_t)$.

%\textit{Transition functions} are:

\textit{Rewards per unit time (utility)} are:
\begin{equation}
U(x) = 
\begin{cases}
\frac{x^{1-\gamma}}{1-\gamma} \text{ for } 0 < \gamma \neq 1 \\
\log(x) \text{ for } \gamma = 1
\end{cases} 
\end{equation}

\textit{Discount factor} is: Relative Risk-Aversion constant
\begin{equation}
	\gamma = -\frac{x \cdot U^{\prime\prime}(x)}{U^\prime(x)}
\end{equation}
\end{solution}

\end{document}