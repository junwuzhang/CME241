\documentclass[12pt]{article}
 
\usepackage[margin=1in]{geometry}
\usepackage{amsmath,amsthm,amssymb}
\usepackage{commath}
\usepackage{graphicx}
\usepackage{float}
\usepackage{caption}
\usepackage{subcaption}
\usepackage{hyperref}
\usepackage{gensymb}
\usepackage{xparse,mathtools}
\usepackage{color,soul}
\usepackage{enumitem}

%%% --------- packages for glossary ------- %%%
\usepackage[acronym,nomain,nonumberlist]{glossaries}
\makeglossaries
%
\newacronym{LHS}{LHS}{left-hand side}
\newacronym{RHS}{RHS}{right-hand side}
\newacronym{CARA}{CARA}{Constant Absolute Risk-Aversion}
\newacronym{CRRA}{CRRA}{Constant Relative Risk-Aversion}

\usepackage[backend=bibtex,style=numeric]{biblatex}
% Select the bibliography file
\addbibresource{references.bib}

\ExplSyntaxOn

\NewDocumentCommand \vect { s o m }
 {
  \IfBooleanTF {#1}
   { \vectaux*{#3} }
   { \IfValueTF {#2} { \vectaux[#2]{#3} } { \vectaux{#3} } }
  ^T
 }

\DeclarePairedDelimiterX \vectaux [1] {\lbrack} {\rbrack}
 { \, \dbacc_vect:n { #1 } \, }

\cs_new_protected:Npn \dbacc_vect:n #1
 {
  \seq_set_split:Nnn \l_tmpa_seq { , } { #1 }
  \seq_use:Nn \l_tmpa_seq { \enspace }
 }
\ExplSyntaxOff
 
\newcommand{\N}{\mathbb{N}}
\newcommand{\R}{\mathbb{R}}
\newcommand{\Z}{\mathbb{Z}}
\newcommand{\Q}{\mathbb{Q}}

\newcommand{\myequation}{\begin{equation}}
\newcommand{\myendequation}{\end{equation}}
\let\[\myequation
\let\]\myendequation
 
\newenvironment{theorem}[2][Theorem]{\begin{trivlist}
\item[\hskip \labelsep {\bfseries #1}\hskip \labelsep {\bfseries #2.}]}{\end{trivlist}}
\newenvironment{lemma}[2][Lemma]{\begin{trivlist}
\item[\hskip \labelsep {\bfseries #1}\hskip \labelsep {\bfseries #2.}]}{\end{trivlist}}
\newenvironment{exercise}[2][Exercise]{\begin{trivlist}
\item[\hskip \labelsep {\bfseries #1}\hskip \labelsep {\bfseries #2.}]}{\end{trivlist}}
\newenvironment{problem}[2][Problem]{\begin{trivlist}
\item[\hskip \labelsep {\bfseries #1}\hskip \labelsep {\bfseries #2.}]}{\end{trivlist}}
\newenvironment{question}[2][Question]{\begin{trivlist}
\item[\hskip \labelsep {\bfseries #1}\hskip \labelsep {\bfseries #2.}]}{\end{trivlist}}
\newenvironment{corollary}[2][Corollary]{\begin{trivlist}
\item[\hskip \labelsep {\bfseries #1}\hskip \labelsep {\bfseries #2.}]}{\end{trivlist}}
\newenvironment{solution}
  {\renewcommand\qedsymbol{$\blacksquare$}\begin{proof}[Solution]}
  {\end{proof}}

\date{Janurary 10, 2020}
 
\begin{document}
 
\title{Homework \#2}
\author{Junwu Zhang\\ 
CME 241: Reinforcement Learning for Finance}
 
\maketitle

\begin{problem}{1}
\text{ }\\
Write out the MP/MRP/MDP/Policy definitions and MRP/MDP Value Function definitions in your own style/notation (so you really internalize these concepts)
\end{problem}
\begin{solution}
\text{ }\\
\begin{itemize}[noitemsep]
	\item MP: Markov Process; a random process where the current state depend only on the most recent state
	\item MRP: Markov Reward Process; a sequence of possible Markov states with values in each state and probabilities from one state to another
	\item MDP: Markov Decision Process; a Markov Reward Process with decisions, in other words, the transition of states does not rely solely on probabilities, rather, they depend on decisions made when being in a state
	\item Policy: a sequence of decisions made in a Markov Decision Process
	\item MRP Value Function: The expected (sum of) return, starting from state s
	in an MDP, and follow the policy $\pi$
	\item MDP Value Function: The expected (sum of) return, starting from state s
	in an MDP, taking action a at each state, and follow the policy $\pi$
\end{itemize}
\end{solution}

\begin{problem}{2}
	\text{ }\\
	The programming assignemnts include:
	\begin{itemize}[noitemsep]
		\item Implement MP/MRP/MDP/Policy definitions in code
		\item Separately implement/convert two reward definitions of MRP
		\item Write code to create a MRP given a MDP and a Policy
		\item Write code for MRP/MDP Bellman Equations
		\item MRP value function
		\item Stationary distribuion of MP
	\end{itemize}
\end{problem}
\begin{solution}
	\text{ }\\
	Code for all problems above is shown in folder \verb|/Assignmennts/0_Basic_DP_RL| in files \verb|mp.py|, \verb|mrp.py|, \verb|policy.py| and \verb|mdp.py|.
\end{solution}

\end{document}