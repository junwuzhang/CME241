\documentclass[12pt]{article}
 
\usepackage[margin=1in]{geometry}
\usepackage{amsmath,amsthm,amssymb}
\usepackage{commath}
\usepackage{graphicx}
\usepackage{float}
\usepackage{caption}
\usepackage{subcaption}
\usepackage{hyperref}
\usepackage{gensymb}
\usepackage{xparse,mathtools}
\usepackage{color,soul}
\usepackage{enumitem}
\usepackage{eufrak}
\usepackage{cleveref}
%\usepackage{datetime}

%%% --------- packages for glossary ------- %%%
\usepackage[acronym,nomain,nonumberlist]{glossaries}
\makeglossaries
%
\newacronym{LHS}{LHS}{left-hand side}
\newacronym{RHS}{RHS}{right-hand side}
\newacronym{CARA}{CARA}{Constant Absolute Risk-Aversion}
\newacronym{CRRA}{CRRA}{Constant Relative Risk-Aversion}
\newacronym{ADP}{ADP}{Approximate Dynamic Programming}
\newacronym{MDP}{MDP}{Markov Decision Process}
\newacronym{MO}{MO}{Market Order}
\newacronym{TOB}{TOB}{Trading Order Book}
\newacronym{LO}{LO}{limit order}
\newacronym{PnL}{PnL}{profit-and-loss}
\newacronym{PG}{PG}{Policy Gradient}
\newacronym{PGT}{PGT}{Policy Gradient Theorem}

\usepackage[backend=bibtex,style=numeric]{biblatex}
% Select the bibliography file
\addbibresource{references.bib}

\ExplSyntaxOn

\NewDocumentCommand \vect { s o m }
 {
  \IfBooleanTF {#1}
   { \vectaux*{#3} }
   { \IfValueTF {#2} { \vectaux[#2]{#3} } { \vectaux{#3} } }
  ^T
 }

\DeclarePairedDelimiterX \vectaux [1] {\lbrack} {\rbrack}
 { \, \dbacc_vect:n { #1 } \, }

\cs_new_protected:Npn \dbacc_vect:n #1
 {
  \seq_set_split:Nnn \l_tmpa_seq { , } { #1 }
  \seq_use:Nn \l_tmpa_seq { \enspace }
 }
\ExplSyntaxOff
 
\newcommand{\N}{\mathbb{N}}
\newcommand{\R}{\mathbb{R}}
\newcommand{\Z}{\mathbb{Z}}
\newcommand{\Q}{\mathbb{Q}}

\newcommand{\myequation}{\begin{equation}}
\newcommand{\myendequation}{\end{equation}}
\let\[\myequation
\let\]\myendequation
 
\newenvironment{theorem}[2][Theorem]{\begin{trivlist}
\item[\hskip \labelsep {\bfseries #1}\hskip \labelsep {\bfseries #2.}]}{\end{trivlist}}
\newenvironment{lemma}[2][Lemma]{\begin{trivlist}
\item[\hskip \labelsep {\bfseries #1}\hskip \labelsep {\bfseries #2.}]}{\end{trivlist}}
\newenvironment{exercise}[2][Exercise]{\begin{trivlist}
\item[\hskip \labelsep {\bfseries #1}\hskip \labelsep {\bfseries #2.}]}{\end{trivlist}}
\newenvironment{problem}[2][Problem]{\begin{trivlist}
\item[\hskip \labelsep {\bfseries #1}\hskip \labelsep {\bfseries #2.}]}{\end{trivlist}}
\newenvironment{question}[2][Question]{\begin{trivlist}
\item[\hskip \labelsep {\bfseries #1}\hskip \labelsep {\bfseries #2.}]}{\end{trivlist}}
\newenvironment{corollary}[2][Corollary]{\begin{trivlist}
\item[\hskip \labelsep {\bfseries #1}\hskip \labelsep {\bfseries #2.}]}{\end{trivlist}}
\newenvironment{solution}
  {\renewcommand\qedsymbol{$\blacksquare$}\begin{proof}[Solution]}
  {\end{proof}}
 
\date{March 04, 2020}
 
\begin{document}
 
\title{Homework \#15}
\author{Junwu Zhang\\ 
CME 241: Reinforcement Learning for Finance \\}
%\date{}
\maketitle

\begin{problem}{1}
\text{ }\\
Write Proof (with precise notation) of the Policy Gradient Theorem
\end{problem}
\begin{solution}
\text{ }\\
For \gls{PG}, we have the following crucial components to the problem:
\begin{itemize}[noitemsep]
	\item \textit{States:} $s_t \in \mathcal{S}$
	\item \textit{Actions:} $a_t \in \mathcal{A}$
	\item \textit{Rewards:} $r_t \in \mathbb{R}, \forall t \in \left\{0, 1, 2, \cdots\right\}$
	\item State Transition Probabilities $\mathcal{P}_{s, s^{\prime}}^{a}=\operatorname{Pr}\left(s_{t+1}=s^{\prime} | s_{t}=s, a_{t}=a\right)$
	\item Initial State Probability Distribution $p_{0}: \mathcal{S} \rightarrow[0,1]$
	\item Expected Rewards $\mathcal{R}_{s}^{a}=E\left[r_{t} | s_{t}=s, a_{t}=a\right]$
	\item Policy Function Approximation $\pi(s, a ; \theta)=\operatorname{Pr}\left(a_{t}=a | s_{t}=s, \theta\right), \theta \in \mathbb{R}^{k}$
	\item \textit{Discount Factor:} $\gamma$
\end{itemize}
The \gls{PGT} can be written as:
\begin{equation}\label{eq:PGT_eqn}
\nabla_{\theta} J(\theta)=\int_{\mathcal{S}} \rho^{\pi}(s) \int_{\mathcal{A}} \nabla_{\theta} \pi(s, a ; \theta) \cdot Q^{\pi}(s, a) \cdot d a \cdot d s
\end{equation}
where $\rho^{\pi}(s)=\int_{\mathcal{S}} \sum_{t=0}^{\infty} \gamma^{t} \cdot p_{0}\left(s_{0}\right) \cdot p\left(s_{0} \rightarrow s, t, \pi\right) \cdot d s_{0}$. We can it \textit{Discounted-Aggregate State-Visitation Measure}, which is a key function for \gls{PG}. $\rho^{\pi}(s)$ depends on $\theta$. We also note that $\nabla_{\theta} \log \pi(s, a ; \theta)$ is the Score Function.

To prove the \gls{PGT}, we can first write:
\begin{align}
J(\theta) &=\int_{\mathcal{S}} p_{0}\left(s_{0}\right) \cdot V^{\pi}\left(s_{0}\right) \cdot d s_{0}\\
&=\int_{\mathcal{S}} p_{0}\left(s_{0}\right) \int_{\mathcal{A}} \pi\left(s_{0}, a_{0} ; \theta\right) \cdot Q^{\pi}\left(s_{0}, a_{0}\right) \cdot d a_{0} \cdot d s_{0}
\end{align}
Then, by calculating $\pi\left(s_{0}, a_{0} ; \theta\right)$ and $Q^{\pi}\left(s_{0}, a_{0}\right)$, we can write $\nabla_{\theta} J(\theta)$ as:
\begin{equation}
\begin{aligned}
\nabla_{\theta} J(\theta) &=\int_{\mathcal{S}} p_{0}\left(s_{0}\right) \int_{\mathcal{A}} \nabla_{\theta} \pi\left(s_{0}, a_{0} ; \theta\right) \cdot Q^{\pi}\left(s_{0}, a_{0}\right) \cdot d a_{0} \cdot d s_{0} \\
&+\int_{\mathcal{S}} p_{0}\left(s_{0}\right) \int_{\mathcal{A}} \pi\left(s_{0}, a_{0} ; \theta\right) \cdot \nabla_{\theta} Q^{\pi}\left(s_{0}, a_{0}\right) \cdot d a_{0} \cdot d s_{0}
\end{aligned}
\end{equation}

Using the Bellman operator, we can expand the action value function $Q^{\pi}\left(s_{0}, a_{0}\right)$ as:
\begin{equation}\label{eq:bellman_expansion}
	Q^{\pi}\left(s_{0}, a_{0}\right) = \mathcal{R}_{s_{0}}^{a_{0}}+\int_{\mathcal{S}} \gamma \cdot \mathcal{P}_{s_{0}, s_{1}}^{a_{0}} \cdot V^{\pi}\left(s_{1}\right) \cdot d s_{1}
\end{equation}
Plugging \Cref{eq:bellman_expansion} into the original equation, we have:
\begin{align}
	\nabla_{\theta} J(\theta)
	&=\int_{\mathcal{S}} p_{0}\left(s_{0}\right) \int_{\mathcal{A}} \nabla_{\theta} \pi\left(s_{0}, a_{0} ; \theta\right) \cdot Q^{\pi}\left(s_{0}, a\right) \cdot d a_{0} \cdot d s_{0} \\
	&+\int_{\mathcal{S}} p_{0}\left(s_{0}\right) \int_{\mathcal{A}} \pi\left(s_{0}, a_{0} ; \theta\right) \cdot \nabla_{\theta}\left(\int_{\mathcal{S}} \gamma \cdot \mathcal{P}_{s_{0}, s_{1}}^{a_{0}} \cdot V^{\pi}\left(s_{1}\right) \cdot d s_{1}\right) \cdot d a_{0} \cdot d s_{0}	
\end{align}
since $\nabla_{\theta} \mathcal{R}_{s_{0}}^{a_{0}}=0$.

Moving $\nabla_{\theta}$ into $\int_{\mathcal{S}}$, and further moving the outside $\int_{\mathcal{A}}$, we have:
\begin{equation}
	\begin{aligned}
	\nabla_{\theta} J(\theta)
	&=\int_{\mathcal{S}} p_{0}\left(s_{0}\right) \int_{\mathcal{A}} \nabla_{\theta} \pi\left(s_{0}, a_{0} ; \theta\right) \cdot Q^{\pi}\left(s_{0}, a\right) \cdot d a_{0} \cdot d s_{0} \\
	&+\int_{\mathcal{S}} p_{0}\left(s_{0}\right) \int_{\mathcal{A}} \pi\left(s_{0}, a_{0} ; \theta\right) \int_{\mathcal{S}} \gamma \cdot \mathcal{P}_{s_{0}, s_{1}}^{a_{0}} \cdot \nabla_{\theta} V^{\pi}\left(s_{1}\right) \cdot d s_{1} \cdot d a_{0} \cdot d s_{0}\\
	&=\int_{\mathcal{S}} p_{0}\left(s_{0}\right) \int_{\mathcal{A}} \nabla_{\theta} \pi\left(s_{0}, a_{0} ; \theta\right) \cdot Q^{\pi}\left(s_{0}, a_{0}\right) \cdot d a_{0} \cdot d s_{0} \\
	&+\int_{\mathcal{S}}\left(\int_{\mathcal{S}} \gamma \cdot p_{0}\left(s_{0}\right) \int_{\mathcal{A}} \pi\left(s_{0}, a_{0} ; \theta\right) \cdot \mathcal{P}_{s_{0, s_{1}}}^{a_{0}} \cdot d a_{0} \cdot d s_{0}\right) \cdot \nabla_{\theta} V^{\pi}\left(s_{1}\right) \cdot d s_{1}
	\end{aligned}
\end{equation}

We can further expand the above equations to:
\begin{equation}
\begin{aligned}
\nabla_{\theta} J(\theta)
&=\int_{\mathcal{S}} p_{0}\left(s_{0}\right) \int_{\mathcal{A}} \cdot \nabla_{\theta} \pi\left(s_{0}, a_{0} ; \theta\right) \cdot Q^{\pi}\left(s_{0}, a_{0}\right) \cdot d a_{0} \cdot d s_{0} \\
&+\int_{\mathcal{S}}\left(\int_{\mathcal{S}} \gamma \cdot p_{0}\left(s_{0}\right) \cdot p\left(s_{0} \rightarrow s_{1}, 1, \pi\right) \cdot d s_{0}\right) \cdot \nabla_{\theta} V^{\pi}\left(s_{1}\right) \cdot d s_{1}
\end{aligned}
\end{equation}
based on the fact that:
\begin{equation}
\int_{\mathcal{A}} \pi\left(s_{0}, a_{0} ; \theta\right) \cdot \mathcal{P}_{s_{0}, s_{1}}^{a_{0}} \cdot d a_{0}=p\left(s_{0} \rightarrow s_{1}, 1, \pi\right)
\end{equation}

Next, expand the above equations to:
\begin{equation}
\begin{aligned}
\nabla_{\theta} J(\theta)
&=\int_{\mathcal{S}} p_{0}\left(s_{0}\right) \int_{\mathcal{A}} \cdot \nabla_{\theta} \pi\left(s_{0}, a_{0} ; \theta\right) \cdot Q^{\pi}\left(s_{0}, a_{0}\right) \cdot d a \cdot d s_{0} \\
&+\int_{\mathcal{S}}\left(\int_{\mathcal{S}} \gamma \cdot p_{0}\left(s_{0}\right) p\left(s_{0} \rightarrow s_{1}, 1, \pi\right) d s_{0}\right) \cdot \nabla_{\theta}\left(\int_{\mathcal{A}} \pi\left(s_{1}, a_{1} ; \theta\right) Q^{\pi}\left(s_{1}, a_{1}\right) d a_{1}\right) d s_{1}
\end{aligned}
\end{equation}
based on the fact that:
\begin{equation}
V^{\pi}\left(s_{1}\right) = \int_{\mathcal{A}} \pi\left(s_{1}, a_{1} ; \theta\right) \cdot Q^{\pi}\left(s_{1}, a_{1}\right) \cdot d a_{1}
\end{equation}

We can then split $\pi Q^\pi$ and calculate the gradient of $Q^\pi$ using Bellman expansion, we have:
\begin{equation}
	\begin{aligned}
	\nabla_{\theta} J(\theta)
	&=\int_{\mathcal{S}} p_{0}\left(s_{0}\right) \int_{\mathcal{A}} \cdot \nabla_{\theta} \pi\left(s_{0}, a_{0} ; \theta\right) \cdot Q^{\pi}\left(s_{0}, a_{0}\right) \cdot d a_{0} \cdot d s_{0} \\
	&+\int_{\mathcal{S}} \int_{\mathcal{S}} \gamma p_{0}\left(s_{0}\right) p\left(s_{0} \rightarrow s_{1}, 1, \pi\right) d s_{0}\left(\int_{\mathcal{A}} \nabla_{\theta} \pi\left(s_{1}, a_{1} ; \theta\right) Q^{\pi}\left(s_{1}, a_{1}\right) d a_{1}+\ldots\right) d s_{1} \\
	&=\sum_{t=0}^{\infty} \int_{\mathcal{S}} \int_{\mathcal{S}} \gamma^{t} \cdot p_{0}\left(s_{0}\right) \cdot p\left(s_{0} \rightarrow s_{t}, t, \pi\right) \cdot d s_{0} \int_{\mathcal{A}} \nabla_{\theta} \pi\left(s_{t}, a_{t} ; \theta\right) \cdot Q^{\pi}\left(s_{t}, a_{t}\right) \cdot d a_{t} \cdot d s_{t}
	\end{aligned}
\end{equation}

Since we know $\int_{\mathcal{A}} \nabla_{\theta} \pi\left(s_{t}, a_{t} ; \theta\right) \cdot Q^{\pi}\left(s_{t}, a_{t}\right) \cdot d a_{t}$ is independent of $t$, we can move $\sum_{t=0}^{\infty}$ inside the integrals and write:
\begin{equation}
	\nabla_{\theta} J(\theta) = \int_{\mathcal{S}} \int_{\mathcal{S}} \sum_{t=0}^{\infty} \gamma^{t} \cdot p_{0}\left(s_{0}\right) \cdot p\left(s_{0} \rightarrow s, t, \pi\right) \cdot ds_{0} \int_{\mathcal{A}} \nabla_{\theta} \pi(s, a ; \theta) \cdot Q^{\pi}(s, a) \cdot da \cdot ds
\end{equation}

Since we also have:
\begin{equation}
	\int_{\mathcal{S}} \sum_{t=0}^{\infty} \gamma^{t} \cdot p_{0}\left(s_{0}\right) \cdot p\left(s_{0} \rightarrow s, t, \pi\right) \cdot d s_{0} \stackrel{\text { def }}{=} \rho^{\pi}(s)
\end{equation}
We can finally write the overall equation as:
\begin{equation}
	\nabla_{\theta} J(\theta)=\int_{\mathcal{S}} \rho^{\pi}(s) \int_{\mathcal{A}} \nabla_{\theta} \pi(s, a ; \theta) \cdot Q^{\pi}(s, a) \cdot d a \cdot d s
\end{equation}
which is the same as \Cref{eq:PGT_eqn}.
\end{solution}

\begin{problem}{2}
	\text{ }\\
	Derive the score function for softmax policy (for finite set of actions)
\end{problem}
\begin{solution}
	Since we know that $\theta$ is $n$-vector and features vector is:
	\begin{equation}
	\phi(s, a)=\left(\phi_{1}(s, a), \ldots, \phi_{n}(s, a)\right) \text { for all } s \in \mathcal{S}, a \in \mathcal{A}
	\end{equation}
	we can weight the actions by doing a linear combination of all the features, namely $\theta^T \cdot \phi(s, a)$.
	Then, we can see that action probabilites are proportional to weights exponentiated, which can be written as:
	\begin{equation}
		\pi(s, a ; \theta)=\frac{e^{\theta^{\top} \cdot \phi(s, a)}}{\sum_{b} e^{\theta^{T} \cdot \phi(s, b)}} \text { for all } s \in \mathcal{S}, a \in \mathcal{A}
	\end{equation}
	By the definition of score function, we can take the logarithm of the above equation, and have:
	\begin{equation}
		\nabla_{\theta} \log \pi(s, a ; \theta)=\phi(s, a)-\sum_{b} \pi(s, b ; \theta) \cdot \phi(s, b)=\phi(s, a)-\mathbb{E}_{\pi}[\phi(s, \cdot)]
	\end{equation}
\end{solution}

\begin{problem}{3}
	\text{ }\\
	Derive the score function for gaussian policy (for continuous actions)
\end{problem}
\begin{solution}
	Slightly differs from the last question, since the action space is now continuous, we have the feature vector as:
	\begin{equation}
	\phi(s)=\left(\phi_{1}(s), \ldots, \phi_{n}(s)\right) \text { for all } s \in \mathcal{S}
	\end{equation}
	Similar to the last question, we have Gaussian Mean as $\theta^T\cdot\phi(s)$ and with the Gaussian policy $a \sim \mathcal{N}\left(\theta^{T} \cdot \phi(s), \sigma^{2}\right)$ for all $s \in \mathcal{S}$, we have the score function for continuous action space as:
	\begin{equation}
		\nabla_{\theta} \log \pi(s, a ; \theta)=\frac{\left(a-\theta^{T} \cdot \phi(s)\right) \cdot \phi(s)}{\sigma^{2}}
	\end{equation}	
\end{solution}

\begin{problem}{4}
	\text{ }\\
	Write code for the REINFORCE Algoithm (Monte-Carlo Policy Gradient Algorithm, i.e., no Critic)
\end{problem}
\begin{solution}
	Code is attached separately.
\end{solution}

\begin{problem}{5}
	\text{ }\\
	Write Proof (with proper notation) of the Compatible Function Approximation Theorem
\end{problem}
\begin{solution}
	The \textit{Compatible Function Approximation Theorem} states:
	If the following two conditions are satisfied:
	\begin{itemize}[noitemsep]
		\item Critic gradient is compatible with the Actor score function
		\begin{equation}
		\nabla_{w} Q(s, a ; w)=\nabla_{\theta} \log \pi(s, a ; \theta)
		\end{equation}
		\item Critic parameters $w$ minimize the following mean-squared error:
		\begin{equation}
		\epsilon=\int_{\mathcal{S}} \rho^{\pi}(s) \int_{\mathcal{A}} \pi(s, a ; \theta)\left(Q^{\pi}(s, a)-Q(s, a ; w)\right)^{2} \cdot d a \cdot d s
		\end{equation}
	\end{itemize}
	Then the Policy Gradient using critic $Q(s, a ; w)$ is exactly:
	\begin{equation}\label{eq:theorem_eqn}
	\nabla_{\theta} J(\theta)=\int_{\mathcal{S}} \rho^{\pi}(s) \int_{\mathcal{A}} \nabla_{\theta} \pi(s, a ; \theta) \cdot Q(s, a ; w) \cdot d a \cdot d s
	\end{equation}
	
	To prove this, we can first use the second point in the theorem and know that:
	\begin{equation}\label{eq:w_transform_1}
	\begin{aligned}
	\int_{\mathcal{S}} \rho^{\pi}(s) \int_{\mathcal{A}} \pi(s, a ; \theta) \cdot\left(Q^{\pi}(s, a)-Q(s, a ; w)\right) \cdot \nabla_{w} Q(s, a ; w) \cdot d a \cdot d s=0 \\
	\forall w \text{ that minimizes } \epsilon=\int_{\mathcal{S}} \rho^{\pi}(s) \int_{\mathcal{A}} \pi(s, a ; \theta) \cdot\left(Q^{\pi}(s, a)-Q(s, a ; w)\right)^{2} \cdot d a \cdot d s
	\end{aligned}
	\end{equation}
	Using the first bullet point of the theorem which is:
	\begin{equation}
	\nabla_{w} Q(s, a ; w)=\nabla_{\theta} \log \pi(s, a ; \theta),
	\end{equation}
	we can replace the subscript and have:
	\begin{equation}\label{eq:w_transform_2}
		\int_{\mathcal{S}} \rho^{\pi}(s) \int_{\mathcal{A}} \pi(s, a ; \theta) \cdot\left(Q^{\pi}(s, a)-Q(s, a ; w)\right) \cdot \nabla_{\theta} \log \pi(s, a ; \theta) \cdot d a \cdot d s=0
	\end{equation}
	Therefore, we can see that \Cref{eq:w_transform_1} and \Cref{eq:w_transform_2} are equal:
	\begin{equation}\label{eq:w_equals}
		\begin{aligned}
		&\int_{\mathcal{S}} \rho^{\pi}(s) \int_{\mathcal{A}} \pi(s, a ; \theta) \cdot Q^{\pi}(s, a) \cdot \nabla_{\theta} \log \pi(s, a ; \theta) \cdot d a \cdot d s \\
		&= \int_{\mathcal{S}} \rho^{\pi}(s) \int_{\mathcal{A}} \pi(s, a ; \theta) \cdot Q(s, a ; w) \cdot \nabla_{\theta} \log \pi(s, a ; \theta) \cdot d a \cdot d s
		\end{aligned}
	\end{equation}	
	Since we also know:
	\begin{equation}\label{eq:w_gradient}
		\nabla_{\theta} J(\theta) =\int_{\mathcal{S}} \rho^{\pi}(s) \int_{\mathcal{A}} \pi(s, a ; \theta) \cdot Q^{\pi}(s, a) \cdot \nabla_{\theta} \log \pi(s, a ; \theta) \cdot d a \cdot d s
	\end{equation}
	Combining \Cref{eq:w_equals} and \Cref{eq:w_gradient}, we have:
	\begin{equation}
	\begin{aligned}
	\nabla_{\theta} J(\theta)
	&=\int_{\mathcal{S}} \rho^{\pi}(s) \int_{\mathcal{A}} \pi(s, a ; \theta) \cdot Q(s, a ; w) \cdot \nabla_{\theta} \log \pi(s, a ; \theta) \cdot d a \cdot d s \\ 
	&=\int_{\mathcal{S}} \rho^{\pi}(s) \int_{\mathcal{A}} \nabla_{\theta} \pi(s, a ; \theta) \cdot Q(s, a ; w) \cdot d a \cdot d s
	\end{aligned}
	\end{equation}
	and we can see that this is the same as \Cref{eq:theorem_eqn}, therefore the theorem is proved.	
\end{solution}

\end{document}