\documentclass[12pt]{article}
 
\usepackage[margin=1in]{geometry}
\usepackage{amsmath,amsthm,amssymb}
\usepackage{commath}
\usepackage{graphicx}
\usepackage{float}
\usepackage{caption}
\usepackage{subcaption}
\usepackage{hyperref}
\usepackage{gensymb}
\usepackage{xparse,mathtools}
\usepackage{color,soul}
\usepackage{enumitem}
\usepackage{eufrak}
\usepackage{cleveref}
%\usepackage{datetime}

%%% --------- packages for glossary ------- %%%
\usepackage[acronym,nomain,nonumberlist]{glossaries}
\makeglossaries
%
\newacronym{LHS}{LHS}{left-hand side}
\newacronym{RHS}{RHS}{right-hand side}
\newacronym{MDP}{MDP}{Markov Decision Process}
\newacronym{MO}{MO}{Market Order}
\newacronym{TOB}{TOB}{Trading Order Book}
\newacronym{LO}{LO}{limit order}
\newacronym{PnL}{PnL}{profit-and-loss}
\newacronym{PG}{PG}{Policy Gradient}
\newacronym{PGT}{PGT}{Policy Gradient Theorem}
\newacronym{VF}{VF}{Value Function}
\newacronym{GLIE}{GLIE}{Greedy in the Limit with Infinite Exploration}

\usepackage[backend=bibtex,style=numeric]{biblatex}
% Select the bibliography file
\addbibresource{references.bib}

\ExplSyntaxOn

\NewDocumentCommand \vect { s o m }
 {
  \IfBooleanTF {#1}
   { \vectaux*{#3} }
   { \IfValueTF {#2} { \vectaux[#2]{#3} } { \vectaux{#3} } }
  ^T
 }

\DeclarePairedDelimiterX \vectaux [1] {\lbrack} {\rbrack}
 { \, \dbacc_vect:n { #1 } \, }

\cs_new_protected:Npn \dbacc_vect:n #1
 {
  \seq_set_split:Nnn \l_tmpa_seq { , } { #1 }
  \seq_use:Nn \l_tmpa_seq { \enspace }
 }
\ExplSyntaxOff
 
\newcommand{\N}{\mathbb{N}}
\newcommand{\R}{\mathbb{R}}
\newcommand{\Z}{\mathbb{Z}}
\newcommand{\Q}{\mathbb{Q}}

\newcommand{\myequation}{\begin{equation}}
\newcommand{\myendequation}{\end{equation}}
\let\[\myequation
\let\]\myendequation
 
\newenvironment{theorem}[2][Theorem]{\begin{trivlist}
\item[\hskip \labelsep {\bfseries #1}\hskip \labelsep {\bfseries #2.}]}{\end{trivlist}}
\newenvironment{lemma}[2][Lemma]{\begin{trivlist}
\item[\hskip \labelsep {\bfseries #1}\hskip \labelsep {\bfseries #2.}]}{\end{trivlist}}
\newenvironment{exercise}[2][Exercise]{\begin{trivlist}
\item[\hskip \labelsep {\bfseries #1}\hskip \labelsep {\bfseries #2.}]}{\end{trivlist}}
\newenvironment{problem}[2][Problem]{\begin{trivlist}
\item[\hskip \labelsep {\bfseries #1}\hskip \labelsep {\bfseries #2.}]}{\end{trivlist}}
\newenvironment{question}[2][Question]{\begin{trivlist}
\item[\hskip \labelsep {\bfseries #1}\hskip \labelsep {\bfseries #2.}]}{\end{trivlist}}
\newenvironment{corollary}[2][Corollary]{\begin{trivlist}
\item[\hskip \labelsep {\bfseries #1}\hskip \labelsep {\bfseries #2.}]}{\end{trivlist}}
\newenvironment{solution}
  {\renewcommand\qedsymbol{$\blacksquare$}\begin{proof}[Solution]}
  {\end{proof}}
 
\date{Februray 14, 2020}
 
\begin{document}
 
\title{Homework \#12}
\author{Junwu Zhang\\ 
CME 241: Reinforcement Learning for Finance \\}
%\date{}
\maketitle

\begin{problem}{1}
\text{ }\\
Prove the Epsilon-Greedy Policy Improvement Theorem
\end{problem}
\begin{solution}
\text{ }\\

The $\epsilon$-Greedy Policy Improvement Theorem states:
\textit{For any $\epsilon$-greedy policy $\pi$, the $\epsilon$-greedy policy $\pi^\prime$ with respect to $q_\pi$ is an improvement, $v_{\pi^{\prime}}(s) \geq v_{\pi}(s)$}

To prove this, we can expand $q_\pi$ and incorporate the $\epsilon$-greedy idea as:
\begin{equation}
\begin{aligned}
q_{\pi}\left(s, \pi^{\prime}(s)\right) &=\sum_{a \in \mathcal{A}} \pi^{\prime}(a | s) q_{\pi}(s, a) \\
&=\epsilon / m \sum_{a \in \mathcal{A}} q_{\pi}(s, a)+(1-\epsilon) \max _{a \in \mathcal{A}} q_{\pi}(s, a)
\end{aligned}
\end{equation}

Expanding the ``greedy'' part of the equation, we have:
\begin{equation}
\begin{aligned}
(1-\epsilon) \max _{a \in \mathcal{A}} q_{\pi}(s, a) \geq (1-\epsilon) \sum_{a \in \mathcal{A}} \frac{\pi(a | s)-\epsilon / m}{1-\epsilon} q_{\pi}(s, a)
\end{aligned}
\end{equation}

Relating it with previous expressions, we have:
\begin{equation}
\begin{aligned}
q_{\pi}\left(s, \pi^{\prime}(s)\right) 
&=\epsilon / m \sum_{a \in \mathcal{A}} q_{\pi}(s, a)+(1-\epsilon) \max _{a \in \mathcal{A}} q_{\pi}(s, a) \\
& \geq \epsilon / m \sum_{a \in \mathcal{A}} q_{\pi}(s, a)+(1-\epsilon) \sum_{a \in \mathcal{A}} \frac{\pi(a | s)-\epsilon / m}{1-\epsilon} q_{\pi}(s, a) \\
&=\sum_{a \in \mathcal{A}} \pi(a | s) q_{\pi}(s, a)=v_{\pi}(s)
\end{aligned}
\end{equation}

Using policy improvement theorem, we can see that $v_{\pi^{\prime}}(s) \geq v_{\pi}(s)$.	
\end{solution}

\newpage
\begin{problem}{2}
	\text{ }\\
	Provide (with clear mathematical notation) the definition of \gls{GLIE}
\end{problem}
\begin{solution}
	\gls{GLIE} states that: 
	\begin{itemize}[noitemsep]
		\item All state-action pairs are explored infinitely many times,
		\begin{equation}
			\lim _{k \rightarrow \infty} N_{k}(s, a)=\infty
		\end{equation}
		\item The policy converges on a greedy policy,
		\begin{equation}
		\lim _{k \rightarrow \infty} \pi_{k}(a | s)=1\left(a=\underset{a^{\prime} \in \mathcal{A}}{\operatorname{argmax}} Q_{k}\left(s, a^{\prime}\right)\right)
		\end{equation}
	\end{itemize}
\end{solution}

\begin{problem}{3}
	\text{ }\\
	Implement the tabular SARSA and tabular SARSA(Lambda) algorithms
\end{problem}
\begin{solution}
	Code is attached separately.
\end{solution}



\end{document}