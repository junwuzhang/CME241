\documentclass[12pt]{article}
 
\usepackage[margin=1in]{geometry}
\usepackage{amsmath,amsthm,amssymb}
\usepackage{commath}
\usepackage{graphicx}
\usepackage{float}
\usepackage{caption}
\usepackage{subcaption}
\usepackage{hyperref}
\usepackage{gensymb}
\usepackage{xparse,mathtools}
\usepackage{color,soul}
\usepackage{enumitem}
\usepackage{eufrak}
\usepackage{cleveref}
%\usepackage{datetime}

%%% --------- packages for glossary ------- %%%
\usepackage[acronym,nomain,nonumberlist]{glossaries}
\makeglossaries
%
\newacronym{LHS}{LHS}{left-hand side}
\newacronym{RHS}{RHS}{right-hand side}
\newacronym{CARA}{CARA}{Constant Absolute Risk-Aversion}
\newacronym{CRRA}{CRRA}{Constant Relative Risk-Aversion}
\newacronym{ADP}{ADP}{Approximate Dynamic Programming}
\newacronym{MDP}{MDP}{Markov Decision Process}
\newacronym{MO}{MO}{Market Order}
\newacronym{TOB}{TOB}{Trading Order Book}
\newacronym{LO}{LO}{limit order}
\newacronym{PnL}{PnL}{profit-and-loss}

\usepackage[backend=bibtex,style=numeric]{biblatex}
% Select the bibliography file
\addbibresource{references.bib}

\ExplSyntaxOn

\NewDocumentCommand \vect { s o m }
 {
  \IfBooleanTF {#1}
   { \vectaux*{#3} }
   { \IfValueTF {#2} { \vectaux[#2]{#3} } { \vectaux{#3} } }
  ^T
 }

\DeclarePairedDelimiterX \vectaux [1] {\lbrack} {\rbrack}
 { \, \dbacc_vect:n { #1 } \, }

\cs_new_protected:Npn \dbacc_vect:n #1
 {
  \seq_set_split:Nnn \l_tmpa_seq { , } { #1 }
  \seq_use:Nn \l_tmpa_seq { \enspace }
 }
\ExplSyntaxOff
 
\newcommand{\N}{\mathbb{N}}
\newcommand{\R}{\mathbb{R}}
\newcommand{\Z}{\mathbb{Z}}
\newcommand{\Q}{\mathbb{Q}}

\newcommand{\myequation}{\begin{equation}}
\newcommand{\myendequation}{\end{equation}}
\let\[\myequation
\let\]\myendequation
 
\newenvironment{theorem}[2][Theorem]{\begin{trivlist}
\item[\hskip \labelsep {\bfseries #1}\hskip \labelsep {\bfseries #2.}]}{\end{trivlist}}
\newenvironment{lemma}[2][Lemma]{\begin{trivlist}
\item[\hskip \labelsep {\bfseries #1}\hskip \labelsep {\bfseries #2.}]}{\end{trivlist}}
\newenvironment{exercise}[2][Exercise]{\begin{trivlist}
\item[\hskip \labelsep {\bfseries #1}\hskip \labelsep {\bfseries #2.}]}{\end{trivlist}}
\newenvironment{problem}[2][Problem]{\begin{trivlist}
\item[\hskip \labelsep {\bfseries #1}\hskip \labelsep {\bfseries #2.}]}{\end{trivlist}}
\newenvironment{question}[2][Question]{\begin{trivlist}
\item[\hskip \labelsep {\bfseries #1}\hskip \labelsep {\bfseries #2.}]}{\end{trivlist}}
\newenvironment{corollary}[2][Corollary]{\begin{trivlist}
\item[\hskip \labelsep {\bfseries #1}\hskip \labelsep {\bfseries #2.}]}{\end{trivlist}}
\newenvironment{solution}
  {\renewcommand\qedsymbol{$\blacksquare$}\begin{proof}[Solution]}
  {\end{proof}}
 
\date{February 04, 2020}
 
\begin{document}
 
\title{Homework \#9}
\author{Junwu Zhang\\ 
CME 241: Reinforcement Learning for Finance \\}
%\date{}
\maketitle

\begin{problem}{1}
\text{ }\\
Write out the full derivation (in LaTeX) of the Avellaneda-Stoikov result we derived in class
\end{problem}
\begin{solution}
\text{ }\\
In this problem, we are still focusing on the optimal market-making problem. One way to solve it can be the method discussed in the paper by Avellanda and Stoikov in 2006. 

The question is formulated as a \gls{MDP}. In discrete-time notation, we can write the \gls{MDP} as:

\textit{States} are ($t, S_t, W_t, I_t$) where $t$ represents the time step, $S_t$ represents the \gls{TOB} mid price at time $t$, $W_t$ is the Market-maker's trading \gls{PnL} at time $t$, and $I_t$ is the Market-maker’s inventory of shares at time $t$. 

\textit{Actions} is $(P_t^{(b)}, N_t^{(b)}, P_t^{(a)}, N_t^{(a)})$

\textit{\gls{TOB} Price Dynamics} is as follows: 
\begin{itemize}[noitemsep]
\item random bid-shares hit $ = X_{t+1}^{(b)} - X_t^{(b)}$ and ask-shares lifted $ = X_{t+1}^{(a)} - X_t^{(a)}$
\item $W_t \rightarrow W_{t+1}$, $I_t \rightarrow I_{t+1}$
\item Stochastic update of $S_t$ to $S_{t+1}$
\end{itemize}

\textit{Reward} at time-step $t+1$ is:
\begin{equation}
R_{t+1}:=\left\{\begin{array}{ll}
0 & \text { for } 1 \leq t+1 \leq T-1 \\
U\left(W_{t+1}+l_{t+1} \cdot S_{t+1}\right) & \text { for } t+1=T
\end{array}\right.
\end{equation}

Therefore, the objective is to find optimal policy $\pi^*(t, S_t, W_t, I_t) = (P_t^{(b)}, N_t^{(b)}, P_t^{(a)}, N_t^{(a)})$ that maximizes $\mathbb{E}[\sum_{t=1}^{T} R_t]$ with discount factor $\gamma=1$. 

Once we have this formulated \gls{MDP}, the discrete-time notation can be converted to the continuous-time setting mentioned in the paper. 

\end{solution}

\end{document}